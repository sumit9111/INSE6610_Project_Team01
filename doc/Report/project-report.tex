\documentclass{easychair}
%\IEEEoverridecommandlockouts
\usepackage{cite}
\usepackage{amsmath,amssymb,amsfonts}
\usepackage{amsmath}
\usepackage{algorithmic}
\usepackage{graphicx, caption}
\usepackage{textcomp}
\usepackage{enumitem}
\usepackage{xcolor}
\usepackage{tabu}
\usepackage{doc}
\usepackage{makeidx}
\usepackage{multicol}
\usepackage{enumerate}
\usepackage{float}
\usepackage{tabularx}
\usepackage{listings}

%%%%%%%%%%%%%% Self Added %%%%%%%%%%%%%
\graphicspath{{./images/}}
\usepackage[utf8]{inputenc}
% \usepackage[english]{babel}

\usepackage{stfloats}
%\usepackage{rsfso}
\usepackage{amsthm}
\usepackage{color,soul}
\newtheorem{lemma}{Lemma}

\newcolumntype{L}[1]{>{\raggedright\arraybackslash}p{#1}}
\newcolumntype{C}[1]{>{\centering\arraybackslash}p{#1}}
\newcolumntype{R}[1]{>{\raggedleft\arraybackslash}p{#1}}

\makeatletter
\newcommand{\thickhline}{%
    \noalign {\ifnum 0=`}\fi \hrule height 1pt
    \futurelet \reserved@a \@xhline
}
\newcolumntype{"}{@{\hskip\tabcolsep\vrule width 0.2pt\hskip\tabcolsep}}

\usepackage{makecell}

%%%%%%%%%%%%%%%%%%%%%%%%%%%%%%%%%%%%%%%%
\def\BibTeX{{\rm B\kern-.05em{\sci\kern-.025emb}\kern-.08emT\kern-.1667em\lower.7ex\hbox{E}\kern-.125emX}}

\begin{document}

\title{DFRWS IoT Forensic Challenge (2018-2019)
Evidence Analysis Report}


\author{
\IEEEauthorblockN{Ali Wazzan}
\IEEEauthorblockA{\textit{CIISE}\\
\textit{Concordia University}\\ Montreal, QC, Canada }\\

\and
\IEEEauthorblockN{Abbas Olfat}
\IEEEauthorblockA{\textit{CIISE}\\
\textit{Concordia University}\\ Montreal, QC, Canada}\\

\and
\IEEEauthorblockN{Ghasem Olfat}
\IEEEauthorblockA{\textit{CIISE}\\
\textit{Concordia University}\\ Montreal, QC, Canada }\\

\and
\IEEEauthorblockN{Isa Abubakar}
\IEEEauthorblockA{\textit{CIISE}\\
\textit{Concordia University}\\ Montreal, QC, Canada }\\

\and
\IEEEauthorblockN{Meryem Ibnoussina }
\IEEEauthorblockA{\textit{CIISE}\\
\textit{Concordia University}\\ Montreal, QC, Canada }\\

\and
\IEEEauthorblockN{Favour Emeakama}
\IEEEauthorblockA{\textit{CIISE}\\
\textit{Concordia University}\\ Montreal, QC, Canada }\\

\and
\IEEEauthorblockN{Sumit Kumar}
\IEEEauthorblockA{\textit{CIISE}\\
\textit{Concordia University}\\ Montreal, QC, Canada }\\

\and
\IEEEauthorblockN{Parisa Ahmadi}
\IEEEauthorblockA{\textit{CIISE}\\
\textit{Concordia University}\\ Montreal, QC, Canada }\\
}
 


\maketitle
% add page number
\thispagestyle{plain}
\pagestyle{plain}



\begin{abstract}
\textbf{The term of IoT means Internets of Things, they are internet connected devices and networks that are used for the sole purpose of automation and monitoring tasks. Technology is growing rapidly and it has became an essential part of our daily life; everyone owns an IoT device that they use for communication which creates and transmit data. This huge amount of data can be  collected and used in a court of law, and this is done by a digital forensic investigators. In this paper, we take on DFRWS IoT Forensic Challenge 2018-2019 challenge and we use efficient approaches that are effectively used for cyber forensics investigations including cloud data, smart things hub, amazon echo devices and automation devices to tackle this challenge.
\textit{Keywords:} Internet of Things, Digital evidence, Computer forensics, Forensic tools.}
\end{abstract}

%\begin{IEEEkeywords}

%\end{IEEEkeywords}

\section{Introduction}
\label{sect:introduction} This report summarizes the result and process of digital forensic analysis on evidences of DFRWS IoT forensic challenge (2018-2019) scenario.

The report is organized as follows: Chapter 1 presents a general overview of the challenge scenario. Chapter 2 presents the tools used for the forensic analysis and describe them. chapter 3 conduct the forensic analysis and investigate using the tools and the data provided by the Challenger. Chapter 4 summarises the paper and what we learned in the course of the challenge.


\subsection{Description of challenge}
\label{callengescenario} The following is given to us by dfrws:
On 17 May 2018 at 10:40, the police were alerted that an illegal drug lab was invaded and
unsuccessfully set on fire. The police respond promptly, and a forensic team is on scene at 10:45,
including a digital forensic specialist.
The owner the illegal drug lab, Jessie Pinkman, is nowhere to be found. Police interrogate two of
Jessie Pinkman’s known associates: D. Pandana and S. Varga. Pandana and Varga admit having
access to the drug lab’s Wi-Fi network but deny any involvement in the raid. They also say that Jessie
Pinkman’s had the IoT security systems installed because he feared attacks from a rival gang and
that Jessie kept the alarm engaged in “Home” mode whenever he was inside the drug lab.
Within the drug lab the digital forensic specialist observes some IoT devices, including an alarm
system (iSmartAlarm), three cameras (QBee Camera, Nest Camera and Arlo Pro) as well as a smoke
detector (Nest Protect). An Amazon Echo and a Wink Hub are also present.
The digital forensic specialist preserves the diagnostic logs from the iSmartAlarm base station, and
acquires a copy of the file system of the Wink Hub. He also collects the iSmartAlarm and Arlo base
stations to perform an in-depth analysis at the forensic laboratory.
The digital forensic specialist also notices that the a QBee Camera seems to be disabled, so he
collects a sample of the network traffic.
Back at the forensic laboratory, the digital forensic specialist uses the bootloader to collect a memory
image of the two base stations as well as an archive of some folder of interest of the Arlo base station.
Jessie Pinkman's Samsung Galaxy S6 Edge is found at the scene, likely dropped during the raid.
The digital forensic specialist acquires a physical image of this Samsung device.


\begin{figure}[h]
    \centering
    \includegraphics[width=0.45\textwidth]{Img/Drug Lab.JPG}
    \caption{Drug Lab}
    \label{fig:Drug Lab}
\end{figure}

\subsection{Challenge Questions}
The Attorney General needs answers to the following questions:
\begin{itemize}
    \item At what time was the illegal drug lab raided?
\item Could any of the two friends of Jessie Pinkman have been involved in the raid?
\item If yes: Which friend?
\item What is the confidence in such hypothesis?
\item How was the QBee camera disabled?
\end{itemize}

\section{Forensic Tools}

\label{FA}In this section we will list and describe the tools used to resolve the challenge.


\subsection{Autopsy} Autopsy is a platform for digital forensics and a graphical interface to The Sleuth Kit and other digital forensics instruments. The law enforcement, military, and corporate examiners use it to determine what has happened on a computer. You can even use this free, fast, easy to use application which is able to analyze all types of mobile devices and digital media.  \cite{pari_1}\cite{pari_2}
\newline Autopsy is the leading open source web forensics which was designed to be intuitive from the box. Installation is simple, and wizards direct you every step of the way. All findings are displayed in a single tree. \cite{pari_3}
\\
We had different types of data gathered from different devices including the smart phone and base stations. So, we had images which should be browsed and information that should be extracted, and to do so Autopsy was one of our tools.

\subsection{Online Json Tool}
JSON (Javascript object notation) is an open standard format which uses text to store and transmit data objects consisting of key-value pairs \cite{Mery_1}.  It is a JSON Editor Online used as a web-based tool to view, edit, format, transform all kind of JSON documents. \\
In this challenge we have data that includes data base files and CSV files extracted from each table of this data base. We find also in the shared drive some JSON files that contain logs and information about voice command data. In our case we used online Json viewer \cite{Mery_2} tool to analyse the JSON logs found in Cloud amazon Database \cite{Mery_3}. in the file echo shared we have All the requests and responses that has been returned via JSON. Figure 1 shows the contents of an Amazon Echo JSON file.


\subsection{Kibana}
Kibana is an open source data visualization dashboard for Elasticsearch. It provides visualization capabilities on top of the content indexed on an Elasticsearch cluster. Users can create bar, line and scatter plots, or pie charts and maps on top of large volumes of data. \cite{Ali_4} \cite{Ali_5} \\ In our case we followed TapiocaPearlo solution in using Kibana after running the python code (parse tools) to visualise, order and classify the data by using to make it readable to the human eye. Using Timelion feature, we were able to filter the unwanted time zones by adjusting the time range.



\subsection{DB Browser for SQLite}
DB Browser for SQLite (DB4S) is a high quality, visual, open source tool to create, design, and edit database files compatible with SQLite. In our case, we used this tool to brows the different databases found in Jessie Pinkman's phone,  Table \ref{tab:DB} is the list of the databases found there. Immense amount of data about Jessie was found, most of them were unrelated to the event but some of them lead us to important evidence that can be used in the case and to track Jessie such as his Amazon account ID, which was found in Amazon databases shown in Figure \ref{fig:Amazon Database} or the name of the operator who disarmed the alarm before the raid happens shown in Figure \ref{fig:ISMD}.

\subsection{Realm Studio}
This extracts the class of database which contains the record about some camera devices, for an instance, here it extracts the data of Arlo devices.\\

\subsection{FTK imager}
FTK imager is a forensic tool which is used for data preview and it is also an imaging tool which allows us to get quick access to the electronic devices so that we can determine the further forensic investigation. The benefit with this tool is that it retrieves the perfect copies (forensic images) from electronics devices without changing the original evidence. FTK uses full of the hardware resources unlike other tools which use partial capacity of it. //  
	
\subsection{Wireshark}
Wireshark is a tool used in digital forensic to capture network traffic and analyze network traffic. These captured network packets are pcap files that when inserted into Wireshark, it is parsed according to traffic packets, packet header information and payload information. Packets captured using Wireshark are saved in .pcap formats ad given for this project.\\
In many cases, Wireshark is also called a packet sniffer. Consider Wireshark as a mail person who takes packages from clients and can scan each package and determines the nature, shape, words, texture of all contents in the packages. Wireshark is dependent on the principles of network connectivity which is the OSI model. The data available for this investigation is the dfrws\_police.pcap. Using Wireshark, I will take note of the nature of the packets captured over the network which includes the size of packets transmitted (Tx) and Received (Rx) and the time the packets were sent.


\subsection{Python}
Python is an interpreted, high-level, general-purpose programming language. Created by Guido van Rossum and first released in 1991, Python's design philosophy emphasizes code readability with its notable use of significant whitespace. \cite{Ali_1} \\
In our case, python was used on jupyter notebook editor \cite{Ali_2} to use the parse tools developed to retrieves the activity records in the database of the devices.


\subsection{Parser Tools}
The following are tools that were developed and used to extract and retrieve activities and data from databases or video frames from the videos. \cite{Ali_3}
\subsubsection{Wink Activity Parser}
python wink\_activity\_parser.py -d \textless DB path\textgreater to run the tool
\subsubsection{Nest Video Recovery Tool}
python nest\_video\_recovery.py -d \textless DB path\textgreater [-o \textless output path\textgreater] [-m] [-a] [-f] to run the tool.
\subsubsection{iSmartAlarm App Dairy Parser}
python ismartalarm\_dairy\_parser.py -d \textless DB path\textgreater to run the tool.
\subsubsection{Server Stream Parser}
python server\_stream\_parser.py -d \textless DB path\textgreater to run the tool.
\subsubsection{Amazon Alexa CIFT Tool Parser}
python alexa\_cift\_parser.py -d \textless DB path\textgreater -t \textless timezone=’UTC+0200’\textgreater to run the tool.

\subsection{Kibana}
Kibana is an open source data visualization dashboard for Elasticsearch. It provides visualization capabilities on top of the content indexed on an Elasticsearch cluster. Users can create bar, line and scatter plots, or pie charts and maps on top of large volumes of data. \cite{Ali_4} \cite{Ali_5} \\ In our case we followed TapiocaPearlo solution in using Kibana after running the python code (parse tools) to visualise, order and classify the data by using to make it readable to the human eye. Using Timelion feature, we were able to filter the unwanted time zones by adjusting the time range. 

\subsection{Network Miner}
Network miner (version 1.6.1) is an open source network analysis tool that works very much like Wireshark. It is particularly a Network Forensic Analysis Tool for Windows but has also been extended to work on other operating systems. Network miner displays the data in a manner that simplifies the analysis process which is time saving for a Forensic Investigator. In our case this tool was used along side Wireshark to assist it in analysis the captured network.

\subsection{Jadx-gui}
jadx-gui is a command line and GUI tools for produce Java source code from Android .apk files. jadx uses dex2jar as a decompiler engine to convert .dex to the  files  which we can review application’s every internal source code structure. It has a built in de-obfuscation plugin with some handy features to convenient analysis such as text search, class search, find function references, etc \cite{Gh_1}  \cite{Gh_2}. 
\subsection{Ghidra}
Ghidra is a software reverse engineering (SRE) framework created and maintained by the National Security Agency Research Directorate. This framework includes a suite of full-featured, high-end software analysis tools that enable users to analyze compiled code on a variety of platforms including Windows, macOS, and Linux. Capabilities include disassembly, assembly, decompilation, graphing, and scripting, along with hundreds of other features. Ghidra supports a wide variety of processor instruction sets and executable formats and can be run in both user-interactive and automated modes. Users may also develop their own Ghidra plug-in components and/or scripts using Java or Python. \cite{Gh_3}
Ghidra supports headless mode, enabling researchers to spin up any number of cloud instances and reverse engineer at scale. Something that would be both technically difficult and very expensive to do in other tools such as IDA Pro. Ghidra can also be deployed in headless mode as a server to enable group collaboration when reverse engineering large binaries. \cite{Gh_4}\cite{Gh_5}

\subsection{Binwalk}
Binwalk, is an open source firmware extraction tool and specifically, designed for identifying and extracting files and code embedded inside of firmware. It’s able to scan a firmware image and search for file signatures to identify and extract filesystem images, executable code, compressed archives, bootloader and kernel images, file formats like JPEGs and PDFs\cite{Ab_1}\cite{Ab_2}. Binwalk is a popular tool to reverse engineer a firmware image to understand how it works and helps to open binaries inside filesystem images to look for vulnerabilities, search for backdoor passwords, digital certificates, hardcoded crypto keys and identify opcodes for a variety of CPU architectures \cite{Ab_3}. In our case binwalk was used for looking into the framework image of iSmartAlarm to find anything important related to the incident, also to find diagnostic  logs related to the door opening and closing before the raid.

\subsection{Volatility}
Volatility Framework is an open source framework implemented in Python under the GNU, for Random Access Memory (RAM) analysis for 32bit and 64bit Windows, Linux, Mac and Android systems The extraction techniques are performed completely independent of the system being investigated but offer visibility into the runtime state of the system \cite{Ab_4}\cite{Ab_5}. Volatility is a single, cohesive framework analyzes RAM dumps from 32- and 64-bit windows, linux, mac, and android systems. It’s Open Source GPLv2, which means you can read it, learn from it, and extend it. it's unparalleled feature sets based on reverse engineering and specialized research. It's comprehensive coverage of file formats – volatility can analyze raw dumps, crash dumps, hibernation files, VMware .vmem, VMware saved state and suspended files (.vmss/.vmsn), VirtualBox core dumps, LiME (Linux Memory Extractor), expert witness (EWF), and direct physical memory over Firewire. Finally and most importantly to our use it is Forensics/IR/malware focus – Volatility was designed by forensics, incident response, and malware experts to focus on the types of tasks these analysts typically form. \\ We used it to analyze Arlo Camera data set that consisted of memory dump image fileNVRAM settings and the TAR archive of the folder /tmp/media/nand.


\newpage
\section{Forensic Analysis of Data}


\subsection{Pinkman's Samsung}
 \subsubsection{Autopsy Findings}
 The Samsung smart phone data was a compressed file which we extracted it with WinRAR application and found 4 files including “blk0\_{sda.bin}”, “blk16\_sdb.bin”, “blk32\_sdc.bin” and “procdata.zip”. 

\begin{itemize}
\itemsep-1.9em
\item blk0\_sda.bin: This is the largest file in image containing the all user and applications data.
\item blk16\_sdb.bin: In Autopsy nothing can be found from this file.
\item blk32\_sdc.bin: In Autopsy nothing can be found from this file.
\item procdata.zip: Contains some information about device hardware like phone model, CPU and disk information, some information about file systems, mounting volume and partitions, etc. 
 \end{itemize}
 
 “blk0\_sda.bin” file which was known as an image file in Autopsy, and when we introduced this file as a data source to the application we could gain access to the volumes of the phone and see different file types including images, videos, audios, data bases, archives, etc.
There were 20 volumes, 2 unallocated volumes and 18 other volumes which their names are mentioned in Table 1.


\begin{table}[h]
  \centering
    \caption{Volumes detected in the ‘blk0\_sda.bin’ file}
    \label{tab:table1}
    \begin{tabular}{|c|c|c|c|c|c|} 
      \hline
BOTA0		& BOTA1		& EFS		& PARAM		& BOOT		& CACHE\\
 \hline
RECOVERY	& OTA		& RADIO		& TOMBSTONES	& DNT		& HIDDEN\\
  \hline
PERSISTENT	& STEADY		& PERSDATA	& SBFS		& SYSTEM		& USERDATA\\
  \hline
    \end{tabular}
\end{table}


\begin{figure}[H]
	\centering
		\includegraphics[width=1.00\textwidth]{autopsy.jpg}
	\caption{Samsung data exploring using Autopsy}
	\label{fig:autopsy}
\end{figure}

By browsing the stored data in these volumes, we could find some useful information in different types including media files (images, screenshots, videos) and databases containing the data of management applications installed on the smart phone related to devices in the lab. 
The extracted traces can be categorized in three categories:
\begin{enumerate}
\item Databases:\\
IoT devices in lab have some applications for managing them installed on the phone and they have some databases, and we could find them in the below mentioned paths.\\
\begin{table}[h]

   
\begin{tabular}{|c|c|m{6cm}|}

\hline
Device & Database & Path  \\
\hline
 iSmartAlarm & iSmartAlarm.DB & /data/iSA.common/databases/\\
\hline                    
 Wink Hub & persistenceDB & /data/com.quirky.android.wink.wink /databases/\\
\hline
Nest Protect/Camera & frame\_database & /data/com.nest.android/cache /f315c6e2b5434a5381f1f5be6f73b4b3/ \\
\hline 
Nest Protect/Camera & cache & /data/data/com.nest.android/databases\\
\hline
 Arlo & default.realm & /data/com.netgear.android/files/\\
\hline
Amazon echo & map\_data\_storage\_v2.db	& 	/data/com.amazon.dee.app/databases/\\
\hline 
\end{tabular}
 \caption{Extracted DBs from Smart Phone}
    \label{tab:DB}
\end{table}

\item Files:\\
In addition to data bases there are some other format of files like XML or json, etc. which may contain some information which we could extract them from the application folder on the smart phone.\\


   
\begin{table}[h]
   
\begin{tabular}{|c|m{5cm}|m{6.3cm}|}
\hline
Device & File & Path\\
\hline
iSmartAlarm & iSmartAlarmData.xml & 	/USERDATA/data/iSA.common
/shared\_prefs/\\
\hline
Wink Hub & wink\_local\_pref\_470654.xml & 	/USERDATA/data/
com.quirky.android.wink.wink
/shared\_prefs/\\
\hline
Wink Hub & user.xml & /USERDATA/data/
com.quirky.android.wink.wink
/shared\_prefs/\\
\hline
Wink Hub & com.quirky.android.wink.wink
\_preferences.xml	& /USERDATA/data
/com.quirky.android.wink.wink
/shared\_prefs/\\
\hline
Nest Protect/Camera	& cache-1332523362.json & /USERDATA/data/com.nest.android
/cache/cache/\\
\hline
Nest Protect/Camera	& com.nest.android.preferences.xml & /USERDATA/data/com.nest.android
/shared\_prefs/\\
\hline
Nest Protect/Camera	& cache-1503821048.json & /USERDATA/data/
com.nest.android/cache/cache/\\
\hline
arlo & Phoenix.xml & /USERDATA/data/com.netgear.android
/shared\_prefs/\\
\hline
QBee Camera & com.vestiacom.qbeecamera
\_preferences.xml & /USERDATA /data/
com.vestiacom. qbeecamera/shared\_prefs\\
\hline
Nest Protect/Camera	& cache-1332523362.json & /USERDATA/data/com.nest.android
/cache/cache/cache-1332523362.json\\
\hline
Amazon echo & service.identity.xml & /USERDATA/data/com.amazon.dee.app
/shared\_prefs/\\
\hline

\end{tabular}
 \caption{Extracted Files from Smart Phone}
   \label{tab:my_label}
\end{table}

\item Media:
The third type of information which was retrieved from the smart phone was the media including both images and videos.\\
\begin{table}[]
\begin{tabular}{|c|c|m{6.3cm}|}							 
\hline
Media owner & Media & path\\
\hline
User & Photographs & /USERDATA/media/0/DCIM/Camera\\
\hline
User & Screenshots & /media/0/DCIM/Screenshots\\
\hline
Nest Cam and Arlo Pro	& JPEG image caches	& /USERDATA/data/
com.quirky.android.wink.wink/cache/
image\_manager\_disk\_cache/\\
\hline
Nest Cam & Cached pictures & /USERDATA/data/
com.nest.android/cache/dcnetwork\\
\hline
Arlo & JPEG image caches & /USERDATA/data/
com.netgear.android/cache/http/\\
\hline
\end{tabular}
 \caption{Extracted Media from Smart Phone}
   \label{tab:my_label}
\end{table}
\end{enumerate}

\textbf{Obtained Results and Evidences:}\\
Finding from Databases and Files would be analyzed in each device section in following. In media part we could find some pictures taken by the user which were related to IoT devices and could give us some information about those devices like their MAC address or serial numbers. These photographs have been stored in ‘USERDATA/media/0/DCIM/Camera’ path as mentioned in Table 4 and gave us the device specification as mentioned below.\\


\begin{table}[h]
\begin{tabular}{|c|c|c|c|}							 
\hline
Pic number	& Device &	Info type &	Info\\
\hline
20180326\_164922.jpg & Arlo Base Station & MAC Address & 08:02:8E:FF:75:4F\\
\hline
20180326\_164922.jpg & Arlo Base Station & Serial Number & 4RD37B75A1EC9\\
\hline
20180410\_091838.jpg & Wink Hub & MAC Address	& B4:79:A7:25:02:FA\\
\hline
20180410\_091838.jpg & Wink Hub & Serial Number & 16170011WZD1\\
\hline
20180410\_091924.jpg & Nest Protect & Network Name & NEST-E345\\
\hline
20180410\_091924.jpg & Nest Protect & Serial Number & 06CA01AC331600CA\\
\hline
20180410\_092015.jpg & iSmartAlarm Base Station & Model Name & iPU3G\\
\hline
20180410\_092015.jpg & iSmartAlarm Base Station & MAC Address & 00:4D:32:09:D9:E4\\
\hline
20180410\_092059.jpg & iSmartAlarm Motion Sensor & Model Name & PIR3G\\
\hline
20180410\_092059.jpg & iSmartAlarm Motion Sensor & Serial Number & 141605015143012\\
\hline
20180410\_092120.jpg & Nest Cam & Model Name & A0005\\
\hline
20180410\_092120.jpg & Nest Cam & MAC Address & 18:B4:30:61:C9:EF\\
\hline
\end{tabular}
 \caption{Obtained Devices Information from Images in Phone}
   \label{tab:my_pari1}
\end{table}

There were also some screenshots taken from the application which gave us some information about other devices as Table \ref{tab:mssdfs} and \ref{tab:my_pari1} illustrates. 
\begin{table}[H]
\begin{tabular}{|c|c|c|c|}							 
\hline
Picture Name	& Device &	Information Type &	Info\\
\hline
Screenshot\_20180502-132904.png & QBee camera & MAC Address & D8:FB:5E:E1:01:92 \\
\hline
\end{tabular}
 \caption{Obtained Devices Information from Screenshots in Phone}
   \label{tab:mssdfs}
\end{table}

\newpage
\subsection{iSmartAlarm Base Station}
iSmartAlarm is a smart home security device do-it-yourself (DIY), operated with a user's smartphone. iSmartAlarm designs and produces the device and appliances.\cite{pari_4}\\
This systems consists of a CubeOne which is called the brain of the system by the company and customers can add different number of Contact Sensors, Motion Sensor or Remote Tags.Contact Sensors can be placed (battery-operated and attached with double-sided tape) at door or window to make customer sure know when someone enters or leaves.\cite{pari_5}\\   
\subsubsection{Memory Image}
\textbf{Obtained result by using Binwalker:}\\
There are two memory image files, 'ismart\_00.img' and 'ismart\_80.img', the two files have the same SHA256 hash value and the 'Compare Files' function in 010 Editor results in all bytes being identical when compared on byte-by-byte basis, they are same file as Figure \ref{fig:ab1} shows. Therefore, the analysis was conducted for ‘ismart\_00.img’ only. 

\begin{figure}[H]
    \centering
    \includegraphics[width=11cm]{Img/Ab1.JPG}
    \caption{Images Hash Values Comparison}
    \label{fig:ab1}
\end{figure}
\newpage
The result of executing the memory image using the tool is summarized in the following table.
\begin{center}

\begin{tabular}{ | m{9cm} | m{4cm}| } 

\hline
PATH & FINDINGS \\ 
\hline
etc\_ro/rcS & gateway : 192.168.1.1,
ethernet : 192.168.1.68
 \\ 
\hline
etc\_ro/Wireless/RT2860AP/RT2860\_default\_vlan
etc\_ro/Wireless/RT2860AP/RT2860\_default\_novlan
 & HostName:ralink 
Login:admin 
Password:admin 
wan\_ip addr:10.10.10.254 
wan\_netmask:255.255.255.0 
wan\_gateway:10.10.10.253 
 \\ 
\hline
sbin/iSmartAlarm.cer & Certificate file
 \\ 
\hline
sbin/log\_pubkey.pem & Public key file
 \\ 
\hline
sbin/logpubkey.pem & Public key file
 \\ 
\hline
usr/share/default\_config & IP Address:192.168.1.68
subnetmark:255.255.255.0
IP Gateway:192.168.1.1
IP DDns:202.99.96.68

 \\ 
\hline

\end{tabular}
\end{center}
\textbf{Diagnostics log:}
There is Diagnostics log file named ‘server\_stream’. the file type is unknown and seemed it’s a custom format file used in iSmartAlarm devices and contains the records between the server and device. By using strings, a tool in linux we extracted all the string records as Figure \ref{fig:ab3} shows.
Sensor Log contains a record of contact sensor and motion sensor. Event classification for each sensor is available in ‘Data Type’ field. When an event occurs, it is stored in the log as ‘ALARMDOOR’ (contact sensor) and ‘ALARMPIR’ (motion sensor), respectively. Value in the ‘Data’ field can be general text or JSON format data. For general text, it is simply used to record the current event, and JSON data is used to send POST request to the cloud service of iSmartAlarm.
\begin{figure}[H]
    \centering
    \includegraphics[width=15cm]{Img/ab3.JPG}
    \caption{Communication Records Between Server and Alarm}
    \label{fig:ab3}
\end{figure}
\newpage
The following Table \ref{tab:'ALARMDOOR’ and ‘ALARMPIR’ } shows the description of the ‘Data’ field in records with ‘ALARMDOOR’ and ‘ALARMPIR’ data type.

\begin{table}[h]
\begin{tabular}{|c|c| m{4cm}|}							 
\hline
Data Type	& Event & Value \\
\hline
ALARMDOOR (Contact Sensor) & F1: Door Open, 
F2: Door Closed
 & F1: door is open, and send to cloud, F2: door is closed, and send to cloud \\
\hline
ALARMPIR (Motion Sensor) & Motion Detected & pir is triggered, and send to cloud \\
\hline

\end{tabular}
 \caption{'ALARMDOOR’ and ‘ALARMPIR’}
   \label{tab:'ALARMDOOR’ and ‘ALARMPIR’ }
\end{table}

The following Table \ref{tab:Diagnostics Log} summarizes only the relevant information on 2018-05-17, the day of the incident. It shows Sensor ID, Device ID, alarm for door, Siren operation, Message and what mode was used.
\begin{table}[h]
\begin{tabular}{|c|c| m{10cm}|}							 
\hline
Time	& SensorID & Description \\
\hline
09:44:53 &000A8540 & one sensor trigger, door is open and send to cloud.
 \\
\hline
09:45:22 & 0006B4E5 & set status to disarm, disarm and change the LED to white breathe.
 \\
\hline
09:47:18 &000A8540  & set status to disarm, disarm and change the LED to white breathe.
 \\
\hline
10:09:57 &000A8540  & set status to disarm, disarm and change the LED to white breathe.
 \\
\hline
10:22:27 & 000A9474 & set status to disarm, disarm and change the LED to white breathe.
 \\
\hline
 10:34:15 &000A8540  & set status to home and door is closed.
 \\
\hline
10:34:19 & 000A9474 & the current alarm sensor is disable in child process, disarm and change the LED to white breathe.
 \\
\hline
10:34:31 & 004D3209D9E4 & receive the command to stop all the resend and alarm command.
 \\
\hline
10:34:36 & 000A8540 & door is open and all the siren need doorbell.
 \\
\hline
10:37:52 & 004D3209D9E4 & receive the command to stop all the resend and alarm command.

 \\
\hline

\end{tabular}
 \caption{Diagnostics Log}
   \label{tab:Diagnostics Log}
\end{table}

\newpage
\textbf{Obtained result by using Autopsy and DB Browser:}\\
There are some snapshots found in the smartphone which were taken from the iSmartAlarm application and shows that the application can shows the status of “The Cube”, Sensors or Remote tags.

\begin{figure}[h]
    \centering
    \includegraphics[width=5cm]{Img/Screenshot20180330203354.png}
    \caption{iSmartAlarm screenshot}
    \label{fig:my_label}
\end{figure}

\begin{figure}[h]
    \centering
    \includegraphics[width=5cm]{Img/Screenshot20180330203515.png}
    \caption{iSmartAlarm Sensors' Status}
    \label{fig:my_label}
\end{figure}

Like other IoT devices this solution has an application installed on the smart phone which stored data on databases and some XML files.
\begin{figure}[h]
    \centering
    \includegraphics[width=10cm]{Img/xml.jpg}
    \caption{ID and Password stored in iSmartAlarm.xml file}
    \label{fig:my_label}
\end{figure}

\subsubsection{Database}
iSmartAlarm database got extracted form ‘/data/iSA.common/databases/’ and then browsed by using DB Browser. According to manual of this device and regarding the status found in the TB\_IPUDiary table in thie database there are four different modes for the iSmartAlarm system including, Arm, Disarm, Panic and Home.\\

\textbf{Arm:}\\
All devices are armed by either Remote Tag or through the App. After the “Arm” is pushed, it takes 90 seconds to allow user(s) to walk out of the premises before the iSmartAlarm system takes effect\cite{pari_9}.\\

\textbf{Disarm:}\\
All devices are disarmed by either Remote Tag or through the App. It allows up to 60 seconds to disarm the system before the siren begins to go off\cite{pari_9}.\\

\textbf{Home:} \\
The contact sensors are armed, and the motion detector is disarmed. After the “Home” is enabled (turned on), it takes 90 seconds to allow user to walk out of the premise before the iSmartAlarm system takes effect\cite{pari_9}.\\

\textbf{Panic:}\\
 A siren is sounded (located in the CubeOne™) at 100dB\cite{pari_9}.\\

In the TB\_IPUDiary table we found a name by ‘pandadodu’ which was the operator who deactivated the alarm system shortly before the man entered the tabletting room at 10:34 on the day of the incident as Figure \ref{fig:ISMD} shows. The nickname 'pandadodu' is probably a combination of the first and last names of him which is D. Pandana.
\begin{figure}[H]
    \centering
    \includegraphics[width=10cm]{Img/iSmartAlarm.JPG}
    \caption{Evidence Found in iSmartDatabase Database}
    \label{fig:ISMD}
\end{figure}


\subsection{Arlo Base Station}

Arlo Technologies is a company known for home automation, producing wireless surveillance cameras. Before its initial public offering on the New York Stock Exchange in August 2018, Arlo had been considered a brand by Netgear.\\

\textbf{Obtained result by using Volatility:}\\
Figure \ref{fig:VD} shows the result of getting info of the image file and we can see here that the image file is not the standard memory dump file and can’t be recognized by Volatility version 2.6.
\begin{figure}[H]
    \centering
    \includegraphics[width=9cm]{Img/Abb2.JPG}
    \caption{Volatility Output }
    \label{fig:VD}
\end{figure}

Therefore we decided to open it with Hex Editor Neo to see the inside, At offset 0x180FCCC, we found a command line string, that contains the name of the memory image file and seems to be used for physical memory acquisition, as shown in Figure \ref{fig:INA}.  The command could be executed in active system by shell interface. Terminal protocol like telnet or SSH or serial interface approached from UART port.

\begin{figure}[H]
    \centering
    \includegraphics[width=13cm]{Img/abb3.JPG}
    \caption{Image Name Appearance }
    \label{fig:INA}
\end{figure}

The memory image seems to be the concatenation of the flash memory and the RAM and many bits of the image seem to be flipped by unknown reason also most area are null-filled or 0xFF-filled have irregularly set bits or unset bits.  We couldn’t find any malicious process which attached in device’s memory.\\\\
\textbf{NAND TAR Archive:}\\
The file ‘arlo\_nand.tar.gz’ contains '/tmp/media/nand' directory of the base station’s active file system. Most are JSON files which contain information about policies, configuration and camera device information and no significant information related to the incident was found in the archive.\\
\textbf{Nvram.log:}\\
This file shows settings of the base station saved in NVRAM and is acquired from the output of command ‘nvram show’ executed by active system and contains various items related to network, user, system parameter, etc., the items are in the form of ‘[key]=[value]’. As Figure \ref{fig:nvram.log} shows.

\begin{figure}[H]
    \centering
    \includegraphics[width=8cm]{Img/abb4.JPG}
    \caption{nvram.log }
    \label{fig:nvram.log}
\end{figure}


\textbf{Obtained result by using Autopsy and DB Browser:}\\

\textbf{Databases:}\\
The realm.default is the Arlo database which was on the smart phone in path USERDATA/data/com.netgear.android/files/.\\

\textbf{Files:}\\
The Phoenix.xml which has been found in the USERDATA/data/com.netgear.android/shared\_prefs/ directory contains some information line User Id and the email of the user.
\begin{figure}[H]
    \centering
    \includegraphics[width=10cm]{Img/phoenix.jpg}
    \caption{phoenix.xml file content}
    \label{fig:my_label}
\end{figure}

\subsection{QBee}
The QBee Camera lets users watch their phone live. When the motion is detected, users can get an alert, watch real-time footage, and sound the built-in siren of the QBee Camera. The QBee Camera reads out the ambient temperature , humidity, light and noise levels with its integrated sensors. \cite{pari_6}\\

\textbf{Obtained result by using Autopsy and DB Browser:}\\

In the /USERDATA/data/com.vestiacom.qbeecamera/shared\_prefs path we can find the com.vestiacom.qbeecamera\_preferences.xml file which contains some encrypted information.\\

QBee cameras are known to have several vulnerabilities. We have experimented directly with the vulnerability CVE-2018-16223 \cite{Gh_6}, which Insecure Cryptographic Storage of credentials in com.vestiacom.qbeecamera\_preferences.xml in the QBee Cam application through 1.0.5 for Android allows an attacker to retrieve the username and password in Figure \ref{fig:Encrypted QBee Account Information}.
\begin{figure}[H]
    \centering
    \includegraphics[width=10cm]{Img/Gh1.JPG}
    \caption{Encrypted QBee Account Information}
    \label{fig:Encrypted QBee Account Information}
\end{figure}

In order to protect user credentials, the application uses the Secure Preferences library to enhance the confidentiality of the storage with the Shared Preferences format. But because of the customization of the library, the data used for encryption can be obtained from the application source. This can be confirmed directly by decompiling the APK by the tools that we’ve mentioned in \ref{fig:Qbee Application APK File}, Figure \ref{fig:Decompile APK with Ghidra and String Search Feature Looking for Password and References} and Figure \ref{fig:Decompile APK with Jadx and String Search Feature Looking for Password and References}.

\begin{figure}[H]
    \centering
    \includegraphics[width=10cm]{Img/Gh2.JPG}
    \caption{Qbee Application APK File}
    \label{fig:Qbee Application APK File}
\end{figure}

\begin{figure}[H]
    \centering
    \includegraphics[width=13cm]{Img/Gh3.JPG}
    \caption{Decompile APK with Ghidra and String Search Feature Looking for Password and References}
    \label{fig:Decompile APK with Ghidra and String Search Feature Looking for Password and References}
\end{figure}

\begin{figure}[H]
    \centering
    \includegraphics[width=13cm]{Img/Gh4.JPG}
    \caption{Decompile APK with Jadx and String Search Feature Looking for Password and References}
    \label{fig:Decompile APK with Jadx and String Search Feature Looking for Password and References}
\end{figure}

\textbf{Three steps of key generation algorithm:}\\
First Getting a “preference key” in Shared preference file (first string value) then divide it into half and put “a!k@ES2,g86AX&D8vn2]” string between them and finally hashing the result with SHA256 hash algorithm, as it's shown in Figure \ref{fig:Key Generation Method in ‘QBee Cam’ Application}.
\begin{figure}[H]
    \centering
    \includegraphics[width=14cm]{Img/gh5.JPG}
    \caption{Key Generation Method in ‘QBee Cam’ Application}
    \label{fig:Key Generation Method in ‘QBee Cam’ Application}
\end{figure}

\textbf{Decryption process:}\\
Encrypted data in Shared preference file need to be decoded using Base64. Then decoded data and the key generated from pervious algorithm are used as input parameters into AES-256-ECB to decrypt username and password. As shown in Figure \ref{fig:Decrypted account information} decrypted QBee camera setting information was obtained by our own experiments. The user name of the QBee camera is “JPinkman” and the password is “Esc\_ioT\_2018”.
\begin{figure}[H]
    \centering
    \includegraphics[width=13cm]{Img/gh6.JPG}
    \caption{Decrypted Account Information}
    \label{fig:Decrypted account information}
\end{figure}

It was possible to access QBee cloud storage with this ID and password, password, but there was no record of the incident. The following figure shows the cloud storage accessed with the Pinkman’s user account in Figure \ref{fig: QBee Cam application access with the Pinkman’s user account}.
\begin{figure}[H]
    \centering
    \includegraphics[width=13cm]{Img/gh7.JPG}
    \caption{QBee Cam Application Access with Pinkman’s User Account}
    \label{fig: QBee Cam application access with the Pinkman’s user account}
\end{figure}

\subsection{Wink Hub}
The Wink Hub allows your vast range of smart products to speak the same wireless language, so you can conveniently monitor them from the Wink app — and customise their interaction. \cite{pari_7}\\

\textbf{Obtained result by using Autopsy and DB Browser:}\\

\textbf{Databases:}\\
Persistence DB in ‘/data/com.quirky.android.wink.wink/databases/’ directory belong to Wink Hub and has 2 table which ‘Elements’ tables contains data about the items and events related to Wink Hub and in json format.\\

\textbf{Files:}\\
The ‘wink\_local\_pref\_470654.xml’ , ‘user.xml’ and ‘com.quirky.android.wink.wink\_preferences.xml’ files in path ‘/USERDATA /data/com.quirky.android.wink.wink/shared\_prefs/’ path respectively contains some information about devices connected to Hub and mobile ID and gmail address.\\


\subsection{Amazon Echo}

Amazon Echo is a hands-free, voice-controlled speaker. To play music, ask questions, make calls, send and receive messages, provide information, news, sports scores, weather, and more, Echo connects to the Alexa Voice Service. Amazon echo is able to use user voice to control compatible smart home devices  such as flipping on the lamp, turning on the coffee maker, or dimming the lights. Echo works with lights, locks, switches, thermostats, and more from WeMo, Philips Hue, ecobee, and Wink.\cite{pari_8}\\

\textbf{Obtained result by using Autopsy and DB Browser:}\\

\textbf{Databases:}\\
map\_data\_storage\_v2.db database located in the ‘USERDATA/data/com.amazon.dee.app/databases/’ path contains an “account” table containing the lab owner account record.\\

\begin{figure}[H]
    \centering
    \includegraphics[width=10cm]{Img/amazon Account.jpg}
    \caption{map\_data\_storage\_v2.db database content}
    \label{fig:my_label}
\end{figure}

\textbf{Files:}\\

The username and the email of owner were found in the service.identity.xml file in ‘/USERDATA /data/com.amazon.dee.app/shared\_prefs/’ directory.\\

\begin{figure}[H]
    \centering
    \includegraphics[width=10cm]{Img/amazon AccountXML.jpg}
    \caption{service.identity.xml file content}
    \label{fig:my_label}
\end{figure}

In PKStorage DB Browser by DB Browser changing the mode of the iSmartAlarm system by the Jessie voice commands was found:\\
\begin{figure}[H]
    \centering
    \includegraphics[width=14cm]{Img/alexa.jpg}
    \caption{Browsing PKStorage database by DB Browser}
    \label{fig:my_label}
\end{figure}
\begin{figure}[H]
    \centering
    \includegraphics[width=14cm]{Img/alexa2.jpg}
    \caption{PKStorage database content}
    \label{fig:my_label}
\end{figure}
\newpage
\textbf{Obtained result by using JSON:}\\
To analyse the Json log we used our tools to be able to extract some useful information such as timestamp, sourceDevices, activityItemData, UUID of the device. Figure \ref{fig:JV} show the same Json logs using online Json viewer. The log is placed in the appendix Figure  \ref{fig:Amazon Echo JSON Log Details}.

\begin{figure}[H]
    \centering
    \includegraphics[width=6cm]{Img/mery2.JPG}
    \caption{Online JSON Viewer Log Details}
    \label{fig:JV}
\end{figure}

In Figure \ref{fig:Timeline of JSon Log} we can see the timeline of the same JSON log.
\begin{figure}[H]
    \centering
    \includegraphics[width=10cm]{Img/mery3.JPG}
    \caption{Timeline of JSON Log }
    \label{fig:Timeline of JSon Log}
\end{figure}

\subsection{Nest Camera and Nest Protect}

The security camera Nest Cam Indoor is a camera for people to keep an eye on what matters, with 24/7 live streaming, a magnetic stand, individual notifications with Nest Aware and one app for all Nest products which make having access from anywhere\cite{pari_10}.\\

Nest Protect is designed to detect carbon monoxide and smoke in a residential place.  It has several sensors to help it understand what's going on in the built place and a photoelectrical sensor to detect sluggish, smouldering fires. Nest Protect can be used as single or multiple alarm stations\cite{pari_11}.\\

\textbf{Obtained result by using Autopsy:}\\
In com.nest.android.preferences.xml file in '/data/com.nest.android/shared\_prefs/' p=directory user email was retrieved.\\

\begin{figure}[H]
    \centering
    \includegraphics[width=10cm]{Img/nest.jpg}
    \caption{com.nest.android.preferences.xml file content}
    \label{fig:my_label}
\end{figure}

\subsection{Network}
In this section, we captured the network traffic using wireshark and network minor the we parse the data specifically according to the behaviour of each device on the network. We looked for the IP addresses with the most interactions between each other and the kind of Layer4 and Layer3 communications used. 
\\
But first, we created a Table of information for the IP addresses, their corresponding MAC addresses. Using the Statistics Option  we took note of the total number of packets to and from each IP address (transmitted and received) and the total number of bytes corresponding to the MAC address.
We noticed that when checking the number of total packets received using the IP address instead of the MAC address, the packets were short by one or two packets. Therefore, the Total Number of Packets in Table \ref{tab:Device Information} are deduced from the wireshark analysis from the MAC address. The next image \ref{fig:wireshark1} shows the number of packets from the statistical analysis of the IP addresses. 

\begin{table}[]
\begin{tabular}{|c|c|c|c|c|}							 
\hline

Device Name	& Device MAC Address &	IP Address & Number of Packets & Bytes\\
\hline

Pinkman’s  SamsungE\_73:e3:78 & 
AC:5F:3E:73:E3:78 & 10.20.30.21 & 129 & 18 \\
\hline
Router - Rasberr\_0e:3b:45 & B8:27:EB:0E:3B:45 & 10.20.30.1 & 4237 & 1642\\
\hline
QBeeCamera & D8:FB:5E:E:1:01:92 & 10.20.30.15	& 115 & 14\\
\hline
Nest Cam – NestLabs\_61:c9:ef & I8:B4:30:61:C9:EF & 10.20.30.13 & 3,924 & 1593\\
\hline
Amazon Echo &  74:75:48:96:23:24 &  10.20.30.23 & 34 &  4977\\
\hline
Nest Labs & 18:B4:30:99:9F:85 & 10.20.30.19 & 6 & 884\\
\hline
\end{tabular}
 \caption{Device Information}
   \label{tab:Device Information}
\end{table}
\textbf{Based on Table \ref{tab:Device Information} - Device Information, the following information can be deduced:} 
\begin{itemize}
\itemsep-0.6em
\item The Router had the most number of packets. It is a reasonable observation being that there are several devices connected to the wifi device, especially IoT devices, as seen in the yellow highlights.
\item The Amazon Echo device has an outrageous amount of total bytes (4977) comparing it to its total number of packets (34) as seen in the red highlights.
\item Most of the communication IP addresses) correspondence with the router are IP addresses within the 10.20.30.0/24 subnet. This means that the devices are within the internal network.
\item The Nest Cam NestLabs device had the second most number of packets as seen in the orange highlights.
\end{itemize}

\begin{figure}[H]
    \centering
    \includegraphics[width=10cm]{Img/wireshark.JPG}
    \caption{com.nest.android.preferences.xml file content}
    \label{fig:wireshark1}
\end{figure}

\textbf{Overall analysis of the dfrw police.pcap file using Network Miner:}
\begin{itemize}
\item There are a total of 29 hosts active on the network at the time of this traffic capture.
\item Two files were found. Filenames google-analytics.com.cer and Google Internet Authority .cer. They both had the same source host [216.58.205.174] and destination host [10.20.30.21] using TLS certificate protocol. 
\item There were no images found on this pcap file.
\item No messages (in cleartext) were found.
\item No credentials were found.
\item No signature-based anomalies (according to NetworkMiner) were found.
\item 17 DNS queries were found.
\item There are 13 sessions (most of which are ssl) in this packet, 10 of which are from Pinkman’s Samsung Phone.
\end{itemize}

The following Figure \ref{fig:Network Miner’s analysis of Sessions} illustrates the network miner analysis of the sessions captured from the network.

\begin{figure}[H]
    \centering
    \includegraphics[width=10cm]{Img/Networkminer.JPG}
    \caption{NetworkMiner’s analysis of Sessions}
    \label{fig:Network Miner’s analysis of Sessions}
\end{figure}

\subsection{Lucid Encoding and State Machine}
One of the asked questions in the challenge was about the probability if being involved someone from inside in this raid and according to the evidences the answer is that Dr. Pandana could be the one who was involved. Because logs in the "TB\_IPUDiary" table of iSmartAlarm table shows that Dr. Pandana was the one who changed the mode of alarm to Disarm before the time that door was opened by raider, which its log was obtained from "server\_stream" file of iSmartAlarm. 
For working on this part, we have a set of observation sequences from the database and another set of observation sequences from the "server\_stream" file and the statemachine for the iSmartAlarm system.\\

\begin{figure}[H]
    \centering
    \includegraphics[width=11cm]{Img/ismartalarm state machine.jpg}
    \caption{iSmartAlarm State Machine}
    \label{fig:iSmartAlarm State Machine}
\end{figure}

\lstset{language=Lucid}
\lstset{frame=lines}
\lstset{caption={Contact Sensor Lucid Encoding}}
\lstset{label={lst:code_direct}}
\lstset{basicstyle=\footnotesize}
\begin{lstlisting}

observation sequence sensor =
	{
		sensor_o_1 ,
		sensor_o_2 ,
		sensor_o_3 ,
		// ...
		sensor_o_22
	};			    
observation sensor_o_1	= ([ SensorID:"000A8540", MessageType :"0",
	TS :"1526290387832", IsAlarm :"1", fileGroup :"0", ProfileId :"2",
	ModeId :"2", DetectAlarm :"0"], 1, 0, 1.0);
observation sensor_o_2	= ([ SensorID:"000A8540", MessageType :"1",
	TS :"1526291071976", IsAlarm :"1", fileGroup :"0", ProfileId :"2",
	ModeId :"2", DetectAlarm :"0"], 1, 0, 1.0);
observation sensor_o_3	= ([ SensorID:"000A8540", MessageType :"0",
	TS :"1526375425600", IsAlarm :"1", fileGroup :"0", ProfileId :"2",
	ModeId :"2", DetectAlarm :"0"], 1, 0, 1.0);
observation sensor_o_4	= ([ SensorID:"000A8540", MessageType :"1",
	TS :"1526377143115", IsAlarm :"1", fileGroup :"0", ProfileId :"2",
	ModeId :"2", DetectAlarm :"0"], 1, 0, 1.0);
observation sensor_o_5	= ([ SensorID:"000A8540", MessageType :"0",
	TS :"1526388088946", IsAlarm :"1", fileGroup :"0", ProfileId :"0",
	ModeId :"0", DetectAlarm :"1"], 1, 0, 1.0);
observation sensor_o_6	= ([ SensorID:"000A8540", MessageType :"1",
	TS :"1526389497550", IsAlarm :"1", fileGroup :"0", ProfileId :"1",
	ModeId :"1", DetectAlarm :"0"], 1, 0, 1.0);
observation sensor_o_7	= ([ SensorID:"000A8540", MessageType :"0",
	TS :"1526389499217", IsAlarm :"1", fileGroup :"0", ProfileId :"1",
	ModeId :"1", DetectAlarm :"1"], 1, 0, 1.0);
observation sensor_o_8	= ([ SensorID:"000A8540", MessageType :"1",
	TS :"1526389865350", IsAlarm :"1", fileGroup :"0", ProfileId :"2",
	ModeId :"2", DetectAlarm :"0"], 1, 0, 1.0);
observation sensor_o_9	= ([ SensorID:"000A8540", MessageType :"0",
	TS :"1526389866297", IsAlarm :"1", fileGroup :"0", ProfileId :"2",
	ModeId :"2", DetectAlarm :"0"], 1, 0, 1.0);
observation sensor_o_10 = ([ SensorID:"000A8540", MessageType :"1",
	TS :"1526389866825", IsAlarm :"1", fileGroup :"0", ProfileId :"2",
	ModeId :"2", DetectAlarm :"0"], 1, 0, 1.0);
observation sensor_o_11 = ([ SensorID:"000A8540", MessageType :"0",
	TS :"1526477537403", IsAlarm :"1", fileGroup :"0", ProfileId :"2",
	ModeId :"2", DetectAlarm :"0"], 1, 0, 1.0);
observation sensor_o_12 = ([ SensorID:"000A8540", MessageType :"1",
	TS :"1526477635447", IsAlarm :"1", fileGroup :"0", ProfileId :"2",
	ModeId :"2", DetectAlarm :"0"], 1, 0, 1.0);
observation sensor_o_13 = ([ SensorID:"000A8540", MessageType :"0",
	TS :"1526477685719", IsAlarm :"1", fileGroup :"0", ProfileId :"2",
	ModeId :"2", DetectAlarm :"0"], 1, 0, 1.0);
observation sensor_o_14 = ([ SensorID:"000A8540", MessageType :"1",
	TS :"1526478437139", IsAlarm :"1", fileGroup :"0", ProfileId :"2",
	ModeId :"2", DetectAlarm :"0"], 1, 0, 1.0);
observation sensor_o_15 = ([ SensorID:"000A8540", MessageType :"0",
	TS :"1526478439855", IsAlarm :"1", fileGroup :"0", ProfileId :"1",
	ModeId :"1", DetectAlarm :"1"], 1, 0, 1.0);
observation sensor_o_16 = ([ SensorID:"000A8540", MessageType :"1",
	TS :"1526478730823", IsAlarm :"1", fileGroup :"0", ProfileId :"2",
	ModeId :"2", DetectAlarm :"0"], 1, 0, 1.0);
observation sensor_o_17 = ([ SensorID:"000A8540", MessageType :"0",
	TS :"1526543093343", IsAlarm :"1", fileGroup :"0", ProfileId :"0",
	ModeId :"0", DetectAlarm :"1"], 1, 0, 1.0);
observation sensor_o_18 = ([ SensorID:"000A8540", MessageType :"1",
	TS :"1526543238967", IsAlarm :"1", fileGroup :"0", ProfileId :"2",
	ModeId :"2", DetectAlarm :"0"], 1, 0, 1.0);
observation sensor_o_19 = ([ SensorID:"000A8540", MessageType :"0",
	TS :"1526544592771", IsAlarm :"1", fileGroup :"0", ProfileId :"0",
	ModeId :"0", DetectAlarm :"1"], 1, 0, 1.0);
observation sensor_o_20 = ([ SensorID:"000A8540", MessageType :"0",
	TS :"1526544592771", IsAlarm :"1", fileGroup :"0", ProfileId :"0",
	ModeId :"0", DetectAlarm :"1"], 1, 0, 1.0);
observation sensor_o_21 = ([ SensorID:"000A8540", MessageType :"1",
	TS :"1526546055471", IsAlarm :"1", fileGroup :"0", ProfileId :"2",
	ModeId :"2", DetectAlarm :"0"], 1, 0, 1.0);
observation sensor_o_22 = ([ SensorID:"000A8540", MessageType :"0",
	TS :"1526546076051", IsAlarm :"1", fileGroup :"0", ProfileId :"2",
	ModeId :"2", DetectAlarm :"0"], 1, 0, 1.0);

\end{lstlisting}

\section{Conclusion}
\label{sect:conclusion} Write conclusion here.


\newpage
\section{Appendix}

\subsection{Figures}
The follow are the figures that supports our report.
\subsubsection{Kibana Data visualisation}
The following Figure \ref{fig:Kibana Data visualisation}, shows the timeline activities of the data taken from the applications.
\begin{figure}[h]
    \centering
    \includegraphics[width=0.92\textwidth]{Img/TimeLion.jpg}
    \caption{Kibana Data visualisation}
    \label{fig:Kibana Data visualisation}
\end{figure}
\subsubsection{Databases}
The following Figure \ref{fig:Amazon Database}, shows Jessie Pinkman's Amazon ID, this ID can be used to track Jessie's activities.
\begin{figure}[h]
    \centering
    \includegraphics[width=0.52\textwidth]{Img/JPID.JPG}
    \caption{Amazon Database}
    \label{fig:Amazon Database}
\end{figure}
\newpage
\subsubsection{QBee}
The following Figure \ref{fig:QBee Network}, shows that on QBee Camera, the TCP 15700 port was opened from which the GET /verifyHTTP request was sent. Addition to that, the highest number of packets were sent between QBee and the router. And the second highest was between QBee and Pinkman’s Samsung Phone.
\begin{figure}[h]
    \centering
    \includegraphics[width=0.92\textwidth]{Img/Minor2.JPG}
    \caption{QBee Network}
    \label{fig:QBee Network}
\end{figure}
\newpage
\subsubsection{Amazon Echo JSON Log}
The following Figure \ref{fig:Amazon Echo JSON Log Details} illustrates Amazon echo log.
\begin{figure}[h]
    \centering
    \includegraphics[width=1.00\textwidth]{Img/Mery1.JPG}
    \caption{Amazon Echo JSON Log Details}
    \label{fig:Amazon Echo JSON Log Details}
\end{figure}
\newpage




\subsection{Codes}
The following are the python codes used to extract and retrieve data.
\subsubsection{Alexa CIFT Database Tool Parser}


\lstset{language=Python}
\lstset{frame=lines}
\lstset{caption={Alexa CIFT Database Tool Parser}}
\lstset{label={lst:code_direct}}
\lstset{basicstyle=\footnotesize}
\begin{lstlisting}
from argparse import ArgumentParser
from datetime import datetime, timezone

from utils.logging.log import Log
from utils.database.misc import dict_factory
from utils.time import to_datetime
from utils.elastic import Elastic

import time
import sqlite3
import os


class CIFTDatabaseParser(object):
    """Parser for parse information from cift database."""

    def __init__(self, database, timezone):
        self.database = database
        self.timezone = timezone

    def parse(self):
        """Parse timeline from database."""
        Log.debug("Extracting activities from database...")

        with sqlite3.connect(self.database) as con:
            con.row_factory = dict_factory
            cur = con.cursor()

            cur.execute("SELECT * FROM TIMELINE")
            activities = cur.fetchall()

        documents = []

        for activity in activities:
            documents.append({
                'time': self.convert_time(activity['date'], activity['time']),
                'user': activity['user'],
                'short': activity['short'],
                'desc': activity['desc'],
                'notes': activity['notes']
            })

        Log.info("Successfully parsed data from database.")

        self.save(documents)

    def save(self, documents):
        with Elastic(index='alexa', doc_type='activity') as elastic:
            elastic.upload(documents, 'time')

        Log.info("Successfully uploaded data into elasticsearch.")

    def convert_time(self, date, _time):
        """Convert splited time into datetime."""
        dtime = datetime.strptime(f"{date} {_time} {self.timezone}", "%Y-%m-%d %H:%M:%S.%f %Z%z")
        
        # check timestamp is 0
        if dtime.timestamp() <= 0:
            return to_datetime(0)

        return dtime.astimezone(timezone.utc)\
                    .replace(tzinfo=None)

    def __del__(self):
        del self


def main(args):
    """Main method for parsing activities."""
    if not os.path.exists(args.database):
        Log.error("cift_amazon_alexa.db file not found.", trace_exc=False)
        return

    cdp = CIFTDatabaseParser(args.database, args.timezone)
    cdp.parse()
    del cdp


if __name__ == '__main__':
    parser = ArgumentParser(description="Amazon Alexa CIFT database parser v1.0")
    parser.add_argument("-d", "--database", dest="database", type=str, required=True,
                        help="cift_amazon_alexa.db database file path")
    parser.add_argument("-t", "--timezone", dest="timezone", type=str, required=True,
                        help="Timezone which used at Amazon Alexa. ex) UTC+2 -> UTC+0200")

    args = parser.parse_args()
    main(args)

\end{lstlisting}


    
\subsubsection{iSmartAlarm Dairy Parser}


\lstset{language=Python}
\lstset{frame=lines}
\lstset{caption={iSmartAlarm Dairy Parser}}
\lstset{label={lst:code_direct}}
\lstset{basicstyle=\footnotesize}
\begin{lstlisting}

from argparse import ArgumentParser

from utils.logging.log import Log
from utils.database.misc import dict_factory
from utils.elastic import Elastic
from utils.time import to_datetime

import sqlite3
import os


class DairyParser(object):
    """Parser for generating timeline from iSmartAlarm Application."""

    def __init__(self, database):
        self.database = database  # iSmartAlarm database

    def parse(self):
        """Parse dairies from database."""
        Log.debug("Extracting diaries from database...")

        tables = {
            'CameraDairy': {
                'name': 'TB_CameraDairy',
            },
            'IPUDairy': {
                'name': 'TB_IPUDairy',
            },
            'ISC3Dairy': {
                'name': 'TB_ISC3Dairy',
            },
            'SensorDairy': {
                'name': 'TB_SensorDairy',
            }
        }

        # get dairies from database
        with sqlite3.connect(self.database) as con:
            con.row_factory = dict_factory
            cur = con.cursor()

            for key in tables.keys():
                name = tables[key]['name']
                cur.execute(f"SELECT * FROM {name}")
                tables[key]['data'] = cur.fetchall()

        Log.debug("Successfully parsed data from database.")

        self.save(tables)

    def save(self, tables):
        """Save history into elasticsearch."""
        for key in tables.keys():
            data = tables[key]['data']
            if data:
                for i in range(len(data)):
                    data[i]['date'] = to_datetime(data[i]['date'])

                with Elastic(index=key.lower(), doc_type=key.lower()) as elastic:
                    elastic.upload(data, 'date')
                
                Log.info(f"Successfully uploaded {key} data into elasticsearch.")

    def __del__(self):
        del self


def main(args):
    """Main method for parsing dairies."""
    if not os.path.exists(args.database):
        Log.error("iSmartAlarm.DB file not found.", trace_exc=False)
        return

    dp = DairyParser(args.database)
    dp.parse()
    del dp


if __name__ == '__main__':
    parser = ArgumentParser(description="iSmartAlarm dairy parser v1.0")
    parser.add_argument("-d", "--database", dest="database", type=str, required=True,
                        help="iSmartAlarm.DB database file path")

    args = parser.parse_args()
    main(args)


\end{lstlisting}

\subsubsection{Nest Video Recovery Tool}

\lstset{language=Python}
\lstset{frame=lines}
\lstset{caption={Nest Video Recovery Tool}}
\lstset{label={lst:code_direct}}
\lstset{basicstyle=\footnotesize}
\begin{lstlisting}

from argparse import ArgumentParser
from datetime import datetime
from pathlib import Path

from utils.logging.log import Log
from utils.elastic import Elastic
from utils.time import to_datetime

import sqlite3
import os


class VideoExtractor(object):
    """Extractor for recovering video frame from Nest Application."""

    def __init__(self, database, output):
        self.database = database
        self.rawvideos = []  # video file list for convert and merge file
        self.videotimes = {}  # video recording time
        self.output = output  # video and log output directory

        # make directory when directory not exist
        if not os.path.exists(output):
            os.mkdir(output)

    def _gen_filename(self, timestamp):
        return datetime.strftime(
            datetime.utcfromtimestamp(timestamp / 1000 + 7200), "%Y-%m-%d_%H:%M:%S")

    def extract(self, merge, frame, add_timeline):
        """Extract frames from database."""
        Log.debug("Extracting videos from database...")

        if frame:
            with sqlite3.connect(self.database) as con:
                cur = con.cursor()
                cur.execute("SELECT frame_time, gop_start_rowid, sps_bytes, pps_bytes, frame_bytes, chunk_complete FROM frame_raw_data_table")
                rows = cur.fetchall()
            sps_bytes = None
            pps_bytes = None
            videobuf = None
            count = 0

            timestamps_by_video = {}
            frames_by_video = {}

            for frame in rows:
                frame_time, gop_start_rowid, _sps_bytes, _pps_bytes, frame_bytes, chunk_complete = frame

                if gop_start_rowid == -1:
                    # set new sps and pps bytes
                    sps_bytes = _sps_bytes
                    pps_bytes = _pps_bytes
                    videobuf = pps_bytes + sps_bytes + frame_bytes
                    timestamps_by_video[count] = [frame_time]
                else:
                    videobuf = videobuf + frame_bytes
                    timestamps_by_video[count].append(frame_time)

                if chunk_complete == 1:
                    frames_by_video[count] = videobuf
                    sps_bytes = None
                    pps_bytes = None
                    videobuf = None
                    count += 1

            if videobuf:
                frames_by_video[count] = videobuf

            for key in frames_by_video.keys():
                # save h264 file
                with open(os.path.join(self.output, f'{key}.h264'), 'wb') as f:
                    f.write(frames_by_video[key])

                i = 0
                for timestamp in timestamps_by_video[key]:
                    os.system(f'ffmpeg -i {self.output}/{key}.h264 -c:v libx264 -filter:v "select=gte(n\,{i})" -frames:v 1 -f h264 {self.output}/{key}_{i}.h264')
                    os.system(f'ffmpeg -i {self.output}/{key}_{i}.h264 -frames:v 1 -f image2 {self.output}/{self._gen_filename(timestamp)}.png')
                    os.remove(f'{self.output}/{key}_{i}.h264')
                    i += 1

                os.remove(f'{self.output}/{key}.h264')
            Log.info(f"Successfully saved image by frame.")

        else:
            with sqlite3.connect(self.database) as con:
                cur = con.cursor()
                cur.execute("SELECT * FROM frame_raw_data_table")
                rows = cur.fetchall()

            videobuf = ""  # temporary buffer for constructing video
            videoname = ""  # name of video file
            count = 0  # video file counter

            for row in rows:
                if row[4]:
                    if videoname:
                        with open(videoname, "wb") as f:
                            f.write(videobuf)
                        self.rawvideos.append(videoname)

                    videobuf = row[5]
                    videobuf += row[4]
                    videobuf += row[6]

                    videoname = os.path.join(self.output, f"{count}.tmp")
                    self.videotimes[videoname] = [row[0]]

                    count += 1
                else:
                    videobuf = videobuf + row[6]

                    if row[0] not in self.videotimes[videoname]:
                        self.videotimes[videoname].append(row[0])

            if videobuf:
                with open(videoname, "wb") as f:
                    f.write(videobuf)
                self.rawvideos.append(videoname)

            Log.info(f"Successfully extrated {count} video files.")

            self.save(merge)

            documents = []

            for filename in self.videotimes.keys():
                runtime = self.videotimes[filename]
                start, end = to_datetime(runtime[0]), to_datetime(runtime[-1])
                filename = os.path.basename(filename).replace('tmp', 'mp4')

                documents.append({
                    'start_time': start,
                    'end_time': end,
                    'filename': filename
                })

            # write history as file
            with open(os.path.join(self.output, 'video_list.txt'), 'w') as f:
                for document in documents:
                    f.write(f"{document['filename']}: {document['start_time']} - {document['end_time']}\n")

            # upload to elasticsearch for add timeline
            if add_timeline:
                with Elastic(index='nest', doc_type='video') as elastic:
                    elastic.upload(documents, 'start_time')

    def save(self, merge, frame):
        """Convert and save into playable video."""
        Log.info("Converting video file codec format...")

        for video in self.rawvideos:
            os.system(f"ffmpeg -f h264 -r 10 -i {video} -c copy {video.split('.')[0]}.mp4")

            # remove original file
            if os.path.exists(video):
                os.remove(video)

        Log.info("Successfully convert the video file codec.")

        if merge:
            Log.info("Merging videos..")

            videos = '|'.join([video.split('.')[0] + ".mp4" for video in self.rawvideos])
            os.system(f"ffmpeg -f concat -i \"concat:{videos}\" -c copy video.mp4")

            for video in self.rawvideos:
                os.remove(f"{video.split('.')[0]}.mp4")

            Log.info(f"Successfully merged {len(self.rawvideos)} videos.")

    def __del__(self):
        del self

def main(args):
    """Main method for recovering video."""
    # if frame_database file not found
    if not os.path.exists(args.database):
        Log.error("frame_database file not found.", trace_exc=False)
        return

    ve = VideoExtractor(args.database, args.output)
    ve.extract(args.merge, args.frame, args.add_timeline)
    del ve


if __name__ == "__main__":
    parser = ArgumentParser(description="Nest video recovery tool v1.0")
    parser.add_argument("-d", "--database", dest="database", type=str, required=True,
                        help="frame_database file path")
    parser.add_argument("-o", "--output", dest="output", type=str, default="output",
                        help="extracted video output file directory")
    parser.add_argument("-m", "--merge", dest="merge", type=bool, default=False,
                        help="merge all frames extracted from database")
    parser.add_argument("-a", "--add-timeline", dest="add_timeline", type=bool, default=False,
                        help="Add recording history at timeline with filename")
    parser.add_argument("-f", "--frame", dest="frame", type=bool, default=False,
                        help="Save by frame as a image")

    args = parser.parse_args()
    main(args)


\end{lstlisting}

\subsubsection{Server Stream Parser}

\lstset{language=Python}
\lstset{frame=lines}
\lstset{caption={Server Stream Parser}}
\lstset{label={lst:code_direct}}
\lstset{basicstyle=\footnotesize}
\begin{lstlisting}

from argparse import ArgumentParser
from collections import namedtuple

from utils.elastic import Elastic
from utils.logging.log import Log
from utils.time import to_datetime

import os
import json
import struct
import sys

class ISADiagnoticsStreamParser:
    def __init__(self, diagnotics_stream):
        try:
            with open(diagnotics_stream, 'rb') as f:
                self.stream = f.read()
        except:
            #alert error. need to logging.
            sys.exit(-1)

    def get_unstructured_log(self):
        self.unstructured_stream_parser(self.stream[0x60000:0xE0000])
        return self.unstructured_log

    def get_sensor_log(self):
        self.sensor_log_stream_parser(self.stream[0xE0000:])
        return self.sensor_log

    def sensor_log_stream_parser(self, stream):
        self.sensor_log = []
        stream = stream.replace(b'\x00', b'')
        stream = stream.split(b'$@')
        for s in stream:
            if not s:
                continue
            _, sign, desc = s.split(b"::")
            if sign == b'ALARMDOOR' or sign == b'ALARMPIR':
                try:
                    data = json.loads(desc)
                    log = self.__sensor_log_repackager(data)
                    self.sensor_log.append(log)
                except:
                    pass

    def unstructured_stream_parser(self, stream):
        self.index = 0
        self.unstructured_log = []
        while True:
            self.log = dict()
            sign = stream.find(b'\x24\x40')
            if sign != -1:
                size = struct.unpack("<L", stream[sign+0x2:sign+0x6])[0]
                parsed_data = stream[sign+0xA:sign+0xA+size]
                parsed_data = parsed_data.split(b'::')
                stream = stream[sign+0xA+size:]
            else:
                break
            self.log.update({
                'idx': self.index,
                'tag1': parsed_data[1][:2].decode(),
                'tag2': parsed_data[1][2:].decode(),
                'size': len(parsed_data[2])
            })
            if len(parsed_data[2]) < 16:
                self.__general_parse(parsed_data[2])
            else:
                self.__isa_parse(parsed_data[2])
            self.__classifier()
            self.unstructured_log.append(self.log)
            self.index += 1

    def __sensor_log_repackager(self, log_data):
        """
            000A8540: Contact Sensor
            0006B4E5: PIR Sensor
        """
        sensors = {
            '000A8540': 'Contact',
            '0006B4E5': 'PIR',
        }
        dt_ = to_datetime(log_data['TS'])
        package = {
            'datetime': dt_,
            'sensor': sensors[log_data['SensorID']],
            'sensor_id': log_data['SensorID'],
            'event': False,
            'siren': False,
            'is_detected': False,
        }
        if log_data['MessageType'] == '0':
            package.update({'event':True})
        if log_data['MessageType'] == '0' and log_data['ModeId'] == '2':
            package.update({'siren':True})
        if log_data['DetectAlarm'] == '1':
            package.update({'is_detected':True})
        return package

    def __unpack_to_update(self, data, labels, unpack_tags):
        pseudo_tag = namedtuple('pseudo_tag', labels)
        structured = pseudo_tag(*struct.unpack(unpack_tags, data))._asdict()
        self.log.update(structured)

    def __isa_parse(self, data):
        data_size = self.log['size'] - 16
        self.__unpack_to_update(data, 'sign type1 type2 type3 data', '<4sLLL'+str(data_size)+'s')
        if self.log['sign'].startswith(b'ISA'):
            self.log.update({'desc': 'isa'})
        else:
            self.log.update({'desc': 'general'})

        self.log.update({'sign': self.log['sign'].decode()})

    def __general_parse(self, data):
        if len(data) % 4 != 0:
            self.log['desc'] = 'dummy'
            return
        self.__unpack_to_update(data, 'type1 type2', '<LL')
        self.log['desc'] = 'general'

    def __classifier(self):
        if self.log['desc'] == 'isa':
            types = [self.log[x] for x in ['type1', 'type2', 'type3']]
            if types == [21, 1, 10]:
                self.log.update({
                    'data_type': 'datetime',
                    'data': to_datetime(self.log['data'])
                })
            else:
                self.log.update({'data_type':'raw'})


def main(args):
    if not os.path.exists(args.server_stream):
        Log.error("server_stream file not exist.")
        return

    Log.info("Start parsing iSmartAlarm diagnotics stream...")

    isap = ISADiagnoticsStreamParser(args.server_stream)
    unstructured_log = isap.get_unstructured_log()
    sensor_log = isap.get_sensor_log()

    with Elastic(index='unstructured_log', doc_type='unstructured_log') as elastic:
        datetime_log = []

        for log in unstructured_log:
            if 'data_type' in log.keys():
                if log['data_type'] == 'datetime':
                    datetime_log.append(log)
        elastic.upload(datetime_log, 'data')

    with Elastic(index='sensor_log', doc_type='sensor_log') as elastic:
        elastic.upload(sensor_log, 'datetime')

    Log.info("Successfully upload server_stream data.")

    del isap


if __name__ == '__main__':
    parser = ArgumentParser(description="iSmartAlarm diagnotics server stream parser")
    parser.add_argument('-s', '--server-stream', type=str,
        help='iSmartAlarm server_stream file path')

    args = parser.parse_args()
    main(args)


\end{lstlisting}

\subsubsection{Wink Activity Parser}

\lstset{language=Python}
\lstset{frame=lines}
\lstset{caption={Wink Activity Parser}}
\lstset{label={lst:code_direct}}
\lstset{basicstyle=\footnotesize}
\begin{lstlisting}

from argparse import ArgumentParser

from utils.logging.log import Log
from utils.database.misc import dict_factory
from utils.elastic import Elastic
from utils.time import to_datetime

import sqlite3
import json
import os


class WinkActivityParser(object):
    """Wink database activity parser."""

    def __init__(self, database):
        self.database = database

    def parse(self):
        """Parse activity from database."""
        Log.debug("Extracting activities from database...")

        with sqlite3.connect(self.database) as con:
            con.row_factory = dict_factory
            cur = con.cursor()

            cur.execute("SELECT * FROM Elements WHERE Type='activity'")
            data = cur.fetchall()

        Log.debug("Successfully parsed activities from database.")

        self.save(data)

    def save(self, data):
        """Save activity into elasticsearch."""
        activities = [json.loads(activity['Json']) for activity in data]

        for i in range(len(activities)):
            activities[i]['created_at'] = to_datetime(activities[i]['created_at'])

        with Elastic(index='wink', doc_type='activity') as elastic:
            elastic.upload(activities, 'created_at')

        Log.info("Successfully uploaded wink activity data into elasticsearch.")

    def __del__(self):
        del self


def main(args):
    """Main method for parsing dairies."""
    if not os.path.exists(args.database):
        Log.error("persistenceDB file not found.", trace_exc=False)
        return

    wap = WinkActivityParser(args.database)
    wap.parse()
    del wap


if __name__ == '__main__':
    parser = ArgumentParser(description="Wink Activity parser v1.0")
    parser.add_argument("-d", "--database", dest="database", type=str, required=True,
                        help="persistenceDB database file path")

    args = parser.parse_args()
    main(args)

\end{lstlisting}

\begin{thebibliography}{1}

\bibitem{pari_1}
"Autopsy" [Online]. Available: https://www.sleuthkit.org/autopsy/

\bibitem{pari_2}
"Autopsy Information" [Online]. Available: https://www.autopsy.com/about/

\bibitem{pari_3}
"Autopsy Provided Features " [Online]. Available: https://www.sleuthkit.org/autopsy/features

\bibitem{pari_4}
"iSmartAlarm" [online]. Available:  https://en.wikipedia.org/wiki/ISmartAlarmcite\_note-3

\bibitem{pari_5}
"Sensors" [online]. Available: https://www.ismartalarm.com/devicesl

\bibitem{pari_6}
"QBee FAQ" [online]. Available:  https://qbeecam.com/home.html\#qbee-app

\bibitem{pari_7}
"Wink Hub2" [online]. Available: https://www.wink.com/products/wink\-hub/

\bibitem{pari_8}
"What is eho?" [online]. Available:https://www.amazon.ca/Echo-2nd-Generation-speaker-Charcoal/dp/B0749ZSPN7

\bibitem{pari_9}
"iSmartAlarm owner manual" [online]. Available: https://images-na.ssl-images-amazon.com/images/I/B1-GfwNmEkS.pdf

\bibitem{pari_10}
"Nest Camera" [online]. Available: https://store.google.com/ca/product/nest\_cam

\bibitem{pari_11}
"Nest Smoke Detector" [online].
https://nest.com/support/images/misc-assets/Nest-Protect-(Wired-120V)-User-s-Guide.pdf


\bibitem{Ali_1}
"Python" [Online]. Available: https://en.wikipedia.org/wiki/Python\_(programming\_language)

\bibitem{Ali_2}
"Jupytor Notebook" [Online]. Available: https://jupyter.org/

\bibitem{Ali_3}
"Parse Tools" [Online]. Available: https://github.com/philgeun/TapiocaPearlo

\bibitem{Ali_4}
"Kibana Definition" [Online]. Available: https://en.wikipedia.org/wiki/Kibana

\bibitem{Ali_5}
"Elastic" [Online]. Available: https://www.elastic.co/kibana

\bibitem{Mery_1}
"JASON " [Online]. Available: https://en.wikipedia.org/wiki/JSON
\bibitem{Mery_2}
"JASON Viewer" [Online]. Available:  http://jsonviewer.stack.hu/
\bibitem{Mery_3}
"Amazon Database" [Online]. Available:  https://drive.google.com/drive/folders/1aXX\_CMR2vY5Vw9f\_naQ0-8bxU\_WZp0Pp


\bibitem{Gh_1}
“Jadx-gui” [online]. Available: https://github.com/skylot/jadx
\bibitem{Gh_2}
”Reverse Engineering Android .apk using Jadx” [online]. Available:   http://nestedif.com/android-security/1-reverse-engineering-android-apk-using-jadx/
\bibitem{Gh_3}
“Ghidra” [online]. Available:  https://github.com/NationalSecurityAgency/ghidra
\bibitem{Gh_4}
“Intro to Reverse Engineering with Ghidra” [online]. Available:  https://medium.com/swlh/intro-to-reverse-engineering-45b38370384
\bibitem{Gh_5}
“Ghidra: Android APK” [online]. Available:  https://www.youtube.com/watch?v=At\_T6riSb9A
\bibitem{Gh_6}
“CVE-2018-16223 Detail” [online]. Available:  https://nvd.nist.gov/vuln/detail/CVE-2018-16223

\bibitem{Ab_1}
“Binwalk” [online]. Available: https://github.com/ReFirmLabs/binwalk 
\bibitem{Ab_2}
“Forensic Binwalk” [online]. Available: https://tools.kali.org/forensics/binwalk
\bibitem{Ab_3}
“Reverse Engineering router Firmware With Binwalk” [online]. Available: https://embeddedbits.org/reverse-engineering-router-firmware-with-binwalk/
\bibitem{Ab_4}
“Firmware Security Analysis Extraction Tool” [online]. Available: https://www.darknet.org.uk/2020/04/binwalk-firmware-security-analysis-extraction-tool/
\bibitem{Ab_5}
“Volatility” [online]. Available: https://github.com/volatilityfoundation/volatility/wiki

\end{thebibliography}

\end{document}